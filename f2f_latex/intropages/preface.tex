\startpreface

This thesis follows the manuscript format. Each of the three within carry a similar structure as is typical of the structure one might produce for an academic journal. The three cords form one strand that benefit from being studied together.

Manuscript 1, “Discerning Hydroclimatic Behavior with a Deep Convolutional Residual Regressive Neural Network”, looks at four United States basins and a single streamflow measurement per basin. It also considers just one neural network architecture called Flux to Flow. Upon further consideration, we have realized that the structure itself is more aptly called dcrrnn and pronounced discern. It is called this for two reasons: 1, because of the nature of a trained mind’s ability to learn how to discern the truth from a flurry of information; and 2, because it is an acronym representing the phrase “deep convolutional residual regressive neural network”. The name dcrrnn was accidental and an obfuscation device just to conceal as manuscript 1 was submitted to a double-blind submission. It was undesirable to share the name Flux to Flow because of our ties to the name on the web already. As the work has progressed, dcrrnn is now understood more appropriately as a very specific neural network construction, whereas F2F encompasses the macroscopy of the work. Though there is only a single architecture used in this manuscript, it is both dcrrnn and Flux to Flow. This study considers about seven years of data, considering a daily time scale.

Manuscript 2, “Deep Convolutional Residual Regressive Neural Networks and Sea Surface Temperatures from Aqua and Argo in the 2000s", focuses in greater detail on a single water-focused essential climate variable (sea surface temperature). The deployed experiments, similar to manuscript 1, features solely the dcrrnn architecture under the name Flux to Flow. The time series studied is relatively short in time, only considering a single year of monthly measurements; however, the size of the geography studied is quite large, considering big pieces of the Atlantic, Pacific, and Indian oceans.

Manuscript 3, “Holistic Water Cycle Analysis via the Confluence of Climate Model, Satellite, Ground Truth, and Machine Learning Signal Processing Technologies: Two North American Transboundary River Watersheds", is best understood as the confluence of manuscripts 1 and 2. We fuse measurements of sea surface temperature with measurements of land flow and create images that do not contain nan values, a sometimes frustrating numerical data structure component. We compare the performance of using these fused images against the original technique from manuscript 1 of simply clipping and z-scoring land surface flows and neglecting the ocean. We look at more output targets per moment than manuscript 1, but fewer than manuscript 2. We also use several different neural network constructions to compare the dcrrnn structure against other simpler neural network structures.
